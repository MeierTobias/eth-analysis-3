\section{Appendix}

\createsectiontoc{}

\subsection{Trigonometry}
\subsubsection{Common angles}
\begin{center}
    \includegraphics*[width=0.7\linewidth]{unit_circle.png}
\end{center}

\subsubsection{Identities}
Given: $n, j \in \mathbb{N}$
\begin{align*}
     & \sin(\pi n)                                                                      & = & \: 0                                                    \\
     & \cos(\pi n)                                                                      & = & \: {(-1)}^n                                             \\
     & \sin(\frac{\pi}{2}n) = \left(\frac{1 + {(-1)}^n}{2}\right){(-1)}^{\frac{n+2}{2}} & = & \begin{cases} 0, &n=2j \\ {(-1)}^j,&=2j+1 \end{cases}   \\
     & \cos(\frac{\pi}{2}n) = \left(\frac{1+{(-1)}^n}{2}\right){(-1)}^{\frac{n}{2}}     & = & \begin{cases} 0, &n=2j+1 \\ {(-1)}^j, &n=2j \end{cases} \\
     & \sin\left(\left(n\pm 1\right)\frac{\pi}{2}\right)                                & = & \pm \cos\left(\frac{n\pi}{2}\right)                     \\
     & \cos\left(\left(n\pm 1\right)\frac{\pi}{2}\right)                                & = & \mp \sin\left(\frac{n\pi}{2}\right)                     \\
     & \sin(90^\circ\pm\alpha) = \sin(\frac{\pi}{2} \pm \alpha)                         & = & \cos(\alpha)                                            \\
     & \cos(90^\circ\pm\alpha) = \cos(\frac{\pi}{2} \pm \alpha)                         & = & \mp\sin(\alpha)                                         \\
     & \sin(180^\circ\pm\alpha) = \sin(\pi \pm \alpha)                                  & = & \mp\sin(\alpha)                                         \\
     & \cos(180^\circ\pm\alpha) = \cos(\pi \pm \alpha)                                  & = & -\cos(\alpha)                                           \\
     & \frac{1}{\sin^2 \alpha}                                                          & = & 1+\cot^2 (\alpha)                                       \\
     & \frac{1}{\cos^2 \alpha}                                                          & = & 1+\tan^2 (\alpha)                                       \\
\end{align*}

\subsubsection{Goniometry}
\begin{align*}
    \sin(x\pm y) & =\sin(x)\cos(y)\pm\cos(x)\sin(y)              \\
    \cos(x\pm y) & =\cos(x)\cos(y)\mp\sin(x)\sin(y)              \\
    \tan(x\pm y) & =\frac{\tan(x)\pm\tan(y)}{1\mp\tan(x)\tan(y)}
\end{align*}
\begin{align*}
    \sin(2x)                     & =2\sin(x)\cos(x)                         \\
    \cos(2x)=\cos^2(x)-\sin^2(x) & =1-2\sin^2(x)=2\cos^2(x)-1               \\
    \tan(2x)                     & =\frac{2\tan(x)}{1-\tan^2(x)}            \\
    \sin(3x)                     & =3\sin(x)-4\sin^3(x)                     \\
    \cos(3x)                     & =4\cos^3(x)-3\cos(x)                     \\
    \tan(3x)                     & =\frac{3\tan(x)-\tan^3(x)}{1-3\tan^2(x)}
\end{align*}
\begin{align*}
    \sin^2\left(\frac x2\right)    & =\frac{1-\cos(x)}{2}                                 \\
    \cos^2\left(\frac x2\right)    & =\frac{1+\cos(x)}{2}                                 \\
    \tan^2\left(\frac{x}{2}\right) & =\frac{1-\cos(x)}{1+\cos(x)}                         \\
    \tan\left(\frac x2\right)      & =\frac{1-\cos(x)}{\sin(x)}=\frac{\sin(x)}{1+\cos(x)}
\end{align*}
\begin{align*}
    \sin(x)\cos(y)      & =\frac12\Bigl[\sin(x+y)~+~\sin(x-y)\Bigr]   \\
    \sin(x)\sin(y)      & =\frac12\Bigl[\cos(x-y)~-~\cos(x+y)\Bigr]   \\
    \cos(x)\cos(y)      & =\frac12\Bigl[\cos(x+y)~+~\cos(x-y)\Bigr]   \\
    \sin(x)\pm\sin(y)   & =2\sin\frac{x\pm y}{2}\cos\frac{x \mp y}{2} \\
    \cos(x)+\cos(y)     & =2\cos\frac{x+y}{2}\cos\frac{x-y}{2}        \\
    \cos(x)-\cos(y)     & =-2\sin\frac{x+y}{2}\sin\frac{x-y}{2}       \\
    \tan(x) \pm \tan(y) & = \frac{\sin(x\pm y)}{\cos(x)\cos(y)}
\end{align*}

\subsubsection{Transformations}
\begin{tabular}{ m{0.15\linewidth} | m{0.22\linewidth} | m{0.23\linewidth} | m{0.2\linewidth} } % chktex -2
                                                    & $\sin$                                         & $\cos$                                         & $\tan$                                         \\
    \hline{} % chktex -2
    $\sin(\alpha)=$                                 & -                                              & $\sqrt{1-\cos^2(\alpha)}$                      & $\frac{\tan(\alpha)}{\sqrt{1+\tan^2(\alpha)}}$ \\
    \hline{} % chktex -2
    $\cos(\alpha)=$                                 & $\sqrt{1-\sin^2(\alpha)}$                      & -                                              & $\frac{1}{\sqrt{1+\tan^2(\alpha)}}$            \\
    \hline{} % chktex -2
    $\tan(\alpha)=$ \newline $\alpha \neq 90^\circ$ & $\frac{\sin(\alpha)}{\sqrt{1-\sin^2(\alpha)}}$ & $\frac{\sqrt{1-\cos^2(\alpha)}}{\cos(\alpha)}$ & -
\end{tabular}

\subsubsection{Euler-Relation}
\begin{align*}
    \sin(x)    & =\frac{1}{2i}(e^{ix}-e^{-ix})             \\
    \cos(x)    & =\frac{1}{2}(e^{ix}+e^{-ix})              \\
    \tan(x)    & =\frac{e^{ix}-e^{-ix}}{i(e^{ix}+e^{-ix})} \\
    \sinh(x)   & =\frac{1}{2}(e^{x}-e^{-x})                \\
    \cosh(x)   & =\frac{1}{2}(e^{x}+e^{-x})                \\
    \tanh(x)   & =\frac{e^{x}-e^{-x}}{(e^{x}+e^{-x})}      \\
    e^{ix}     & = \cos(x) + i \cdot \sin(x)               \\
    e^{k\pi i} & = \begin{cases}
                       1  & k = 2n     \\
                       -1 & k = 2n + 1
                   \end{cases} \quad n \in \mathbb{N}      \\
\end{align*}

\subsubsection{Inverse functions}
\begin{align*}
    \sin(\arccos(x))         & = \sqrt{1-x^2}                                                 \\
    \cos(\arcsin(x))         & =\sqrt{1-x^2}                                                  \\
    \sin(\arctan(x))         & =\frac{x}{\sqrt{x^2+1}}                                        \\
    \cos(\arctan(x))         & =\frac{1}{\sqrt{x^2+1}}                                        \\
    \tan(\arcsin(x))         & =x \cdot {(1-x)}^{-\frac{1}{4}}                                \\
    \tan(\arccos(x))         & =x^{-1} \cdot {(1-x)}^{\frac{1}{4}}                            \\
    arsinh(x)                & =\ln(x+\sqrt{x^2+1})                                           \\
    arcosh(x)                & =\ln(x+\sqrt{x^2-1})                                           \\
    artanh(x)                & =\frac{1}{2}\cdot \ln(\frac{1+x}{1-x}); \quad \vert x\vert < 1 \\
    arcoth(x)                & =\frac{1}{2}\cdot \ln(\frac{1+x}{x-1}); \quad \vert x\vert > 1 \\
    \sinh(2 \cdot arsinh(x)) & =2x\sqrt{x^2-1}                                                \\
    \sinh(arcosh(x))         & =\sqrt{x^2-1};\qquad x>0                                       \\
    \cosh(arsinh(x))         & =\sqrt{x^2+1}
\end{align*}


\subsection{Integral calculus}

\subsection{Gamma Functions}
Helpful for removing factorials:
\begin{align*}
    \Gamma(\alpha)             & =\int\limits_{0}^{+\infty}x^{\alpha-1}e^{-x}dx,\quad\alpha>0. \\
    \Gamma\left(\frac12\right) & =\int_0^{+\infty}t^{-1/2}e^{-t}dt=\sqrt{\pi}                  \\ \\
    \Gamma(\alpha+1)           & =\alpha\Gamma(\alpha),\quad \Gamma(1)=1                       \\ \\
    \Gamma(n)                  & =(n-1)!
\end{align*}

\subsection{Dirichlet Superposition}\label{diri_superpos}
\includegraphics*[width = 0.8\linewidth]{dirichlet_superposition.png}\\
\textbf{Superposition}
\begin{equation*}
    u = u_1+u_2+u_3+u_4
\end{equation*}
\textbf{Solution for A}
\begin{align*}
    u_1(x,y) & =\sum_{n=1}^\infty A_n\sin\left(\frac{n\pi x}a\right)\sinh\left(\frac{n\pi(b-y)}a\right)             \\
    A_n      & =\frac2{a\sinh\left(\frac{n\pi b}a\right)}\int_0^{a}f_1(x)\sin\left(\frac{n\pi x}a\right)\mathrm{d}x
\end{align*}
\textbf{Solution for B}
\begin{align*}
    u_2(x,y) & =\sum_{n=1}^\infty B_n\sin\left(\frac{n\pi x}a\right)\sinh\left(\frac{n\pi y}a\right)                \\
    B_n      & =\frac2{a\sinh\left(\frac{n\pi b}a\right)}\int_0^{a}f_2(x)\sin\left(\frac{n\pi x}a\right)\mathrm{d}x
\end{align*}
\textbf{Solution for C}
\begin{align*}
    u_3(x,y) & =\sum_{n=1}^\infty C_n\sinh\left(\frac{n\pi(a-x)}a\right)\sin\left(\frac{n\pi y}b\right)             \\
    C_n      & =\frac2{b\sinh\left(\frac{n\pi a}b\right)}\int_0^{b}g_1(y)\sin\left(\frac{n\pi y}b\right)\mathrm{d}y
\end{align*}
\textbf{Solution for D}
\begin{align*}
    u_4(x,y) & =\sum_{n=1}^\infty D_n\sinh\left(\frac{n\pi x}b\right)\sin\left(\frac{n\pi y}b\right)                \\
    D_n      & =\frac2{b\sinh\left(\frac{n\pi a}b\right)}\int_0^{b}g_2(y)\sin\left(\frac{n\pi y}b\right)\mathrm{d}y
\end{align*}

\subsection{Partial Fraction Decomposition}
\subsubsection{General Method}
Depending on the type and order of the fraction, solve one of the following by hand or by \textit{Gauss elimination}:
\begin{align*}
    \frac{px^2+qx+r}{(x-a)(x-b)(x-c)}     & \overset{!}{=} \frac{A}{x-a}+\frac{B}{x-b}+\frac{C}{x-c}       \\\\
    \frac{px^2+qx+r}{{(x-a)}^2(x-b)}      & \overset{!}{=} \frac{A}{x-a}+\frac{B}{{(x-a)}^2}+\frac{C}{x-b} \\\\
    \frac{px^{2}+qx+r}{(x-a)(x^{2}+bx+c)} & \overset{!}{=} \frac A{x-a}+\frac{Bx+C}{x^2+bx+c}
\end{align*}

\subsubsection{Residue Method}
\textbf{Relative order $>0$}
\begin{align*}
    f(x) & =\frac{px+q}{(x-a)(x-b)} \overset{!}{=} \frac{A}{x-a}+\frac{B}{x-b} \\\\
    A    & =(x-a)\cdot f(x)\Big|_{x=a}                                         \\
    B    & =(x-b)\cdot f(x)\Big|_{x=b}
\end{align*}\\
\textbf{Relative order $=0$}
\begin{align*}
    f(x) & =\frac{px^2+qx+r}{(x-a)(x-b)} \overset{!}{=} A+\frac{B}{x-a}+\frac{C}{x-b} \\\\
    B    & =(x-a)\cdot f(x)\Big|_{x=a}                                                \\
    C    & =(x-b)\cdot f(x)\Big|_{x=b}                                                \\
    A    & = f(x)-\frac{B}{x-a}+\frac{C}{x-b}
\end{align*}\\
These methods can be adapted to all of the types of fractions mentioned in the previous subsection.
