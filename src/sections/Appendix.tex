\section{Appendix}
\subsection{Goniometry}
\begin{align*}
    \sin(x\pm y)&=\sin(x)\cos(y)\pm\cos(x)\sin(y)\\
    \cos(x\pm y)&=\cos(x)\cos(y)\mp\sin(x)\sin(y)\\
    \tan(x\pm y)&=\frac{\tan(x)\pm\tan(y)}{1\mp\tan(x)\tan(y)}
\end{align*}
\begin{align*}
    \sin(2x)&=2\sin(x)\cos(x)\\
    \cos(2x)&=\cos^2(x)-\sin^2(x)=1-2\sin^2(x)=2\cos^2(x)-1
\end{align*}
\begin{align*}
    \cos^2\left(\frac x2\right)&=\frac{1+\cos(x)}{2}\\
    \sin^2\left(\frac x2\right)&=\frac{1-\cos(x)}{2}\\
    \tan\left(\frac x2\right)&=\frac{1-\cos(x)}{\sin(x)}=\frac{\sin(x)}{1+\cos(x)}
\end{align*}
\begin{align*}
    \sin(x)\cos(y)&=\frac12\Bigl[\sin(x+y)~+~\sin(x-y)\Bigr]\\
    \sin(x)\sin(y)&=\frac12\Bigl[\cos(x-y)~-~\cos(x+y)\Bigr]\\
    \cos(x)\cos(y)&=\frac12\Bigl[\cos(x+y)~+~\cos(x-y)\Bigr]
\end{align*}
\subsection{Identities}
Given: $n \in \mathbb{N}$
\begin{align*}
    sin(\pi n) = 0 \; ;  \; cos(\pi n)=(-1)^n
\end{align*}