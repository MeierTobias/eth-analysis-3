\section{Partial Differential Equations}
%
\subsection{Definition}
A partial differential equation (PDE) is an equation involving an \textbf{unknown function} (in contrast to ODEs only one) and some of its \textbf{partial derivatives}.
\begin{enumerate}
    \item A PDE is called \textbf{linear} if it is linear in the unknown function u and its derivatives.
    \item A linear PDE is called \textbf{homogeneous} if each of its terms contains either the function or one of its derivatives.
    \item The \textbf{order} of a PDE is the order of the highest derivative in the PDE.\@
\end{enumerate}
%
%
\subsubsection{Basic Types of PDEs}
\begin{itemize}
    \item 1-dimensional Wave Equation:
    \begin{equation}
        \frac{\partial ^2 u}{\partial t^2} = c^2\frac{\partial u}{\partial x^2}
    \end{equation}
    (\textit{linear, 2nd order, homogeneous, hyperbolic})
    \item 1-dimensional Heat Equation:
    \begin{equation}
        \frac{\partial u}{\partial t} = c^2 \frac{\partial ^2 u}{\partial x^2}
    \end{equation}
    (\textit{linear, 2nd order, homogeneous, parabolic})
    \item 2-dimensional Laplace Equation:
    \begin{equation}
        \nabla^2 u=\frac{\partial^2 u}{\partial x^2}+\frac{\partial ^2 u}{\partial y^2}=0
    \end{equation}
    (\textit{linear, 2nd order, homogeneous, elliptic})
    \item 2-dimensional Poisson Equation:
    \begin{equation}
        \nabla^2 u=\frac{\partial^2 u}{\partial x^2}+\frac{\partial^2 u}{\partial y^2}=f(x,y)
    \end{equation}
    (\textit{linear, 2nd order, inhomogeneous, elliptic})
    \item 2-dimensional Wave Equation:
    \begin{equation}
        \frac{\partial^2u}{\partial t^2}=c^2 \Bigr(\frac{\partial^2 u}{\partial x^2}+\frac{\partial ^2 u}{\partial y^2}\Bigl)=c^2\nabla^2 u
    \end{equation}
    (\textit{linear, 2nd order, homogeneous, hyperbolic})
    \item 2-dimensional Heat Equation
    \begin{equation}
        \frac{\partial u}{\partial t}=c^2\Bigr(\frac{\partial^2 u}{\partial x^2}+\frac{\partial ^2 u}{\partial y^2}\Bigl)=c^2\nabla^2 u
    \end{equation}
    (\textit{linear, 2nd order, homogeneous, parabolic})
    \item 3-dimensional Laplace Equation
    \begin{equation}
        \nabla^2 u=\frac{\partial^2 u}{\partial x^2}+\frac{\partial ^2 u}{\partial y^2}+\frac{\partial ^2 u}{\partial z^2}=0
    \end{equation}
    (\textit{linear, 2nd order, homogeneous, elliptic})
\end{itemize}
%
%
\subsubsection{Solutions of linear PDEs}
\begin{itemize}
    \item Many functions of a certain form can satisfy the PDE
    \item Uniqueness of solutions is achieved by \textbf{boundary} conditions and \textbf{initial} conditions
\end{itemize}
\vspace{5pt}
\textbf{Superposition Principle}\\
If $u_{1}$ and $u_{2}$ are solutions of a \textbf{homogeneous} (linearity is not enough) PDE, then $\alpha u_{1} + \beta u_{2}$ is also a solution of the same PDE $\forall \alpha , \beta \in \mathbb{R}$
%
%
%
\subsection{2nd Order Linear PDEs}
\subsubsection{General Form of 2nd Order Linear PDE}
A 2nd order linear PDE can be written in the form
\begin{equation} Au_{xx}+2Bu_{xy}+Cu_{yy}=F(x,y,u,u_x,u_y)\end{equation}
%
\subsubsection{Classification by Discriminant}
A 2nd order linear PDE has the discriminant $AC-B^2$ and is called
\begin{enumerate}
    \item \textbf{hyperbolic} (wave), if $AC-B^2<0$
    \item \textbf{parabolic} (heat), if $AC-B^2=0$
    \item \textbf{elliptic} (laplace), if $AC-B^2>0$
\end{enumerate}
\vspace{5pt}
Remarks:
\begin{itemize}
    \item Classification is independent of the PDEs dimension\vspace{5pt}
    \item The type of a PDE depends only on the terms of second order: \textbf{Poisson} equations are also \textbf{elliptic}.
\end{itemize}
%
%
%
\subsection{1D Wave Equation - Fourier Series}
To find $u(x,t)$ satisfying the \textbf{homogeneous} 1D wave equation on $x\in[0,L]$ with \textbf{initial} and 
\textbf{boundary} conditions:
\begin{equation*}
    \begin{cases} 
        u_{tt}=c^2u_{xx} & \text{PDE}\\
        u(0,t)=u_0 & \text{IC}\\ 
        u(L,t)=u_L & \text{IC}\\ 
        u(x,0)=f(x)& \text{BC}\\ 
        u_t(x,0)=g(x)& \text{BC}
    \end{cases}
\end{equation*}\\
three steps have to be taken:
\begin{itemize}
    \item[(i)] Separation of variables
    \item[(ii)] Finding the general solution
    \item[(iii)] Finding one solution satisfying IC \& BC
\end{itemize}
%
%

\subsubsection{(i) Separation of Variables}
\begin{align*}
    u(x,t)&=F(x)G(t)\\
    \underbrace{F\ddot{G}}_{u_{tt}}&=c^2\underbrace{F''G}_{u_{xx}}\\
    \frac{F''}{F}=\frac{\ddot{G}}{c^2G}&=k\hspace*{10pt}\Rightarrow\hspace*{10pt}
    \begin{cases} 
        F''=kF\\
        \ddot{G}=c^2kG
    \end{cases}
\end{align*}
%
%
\subsubsection{(ii) General Solution}
By inserting the initial conditions into the separated equation
\begin{equation*}
    \frac{F^{\prime\prime}(x)}{F(x)}=\frac{\ddot{G}(t)}{G(t)}=k\hspace*{10pt}\Rightarrow\hspace*{10pt}
    \begin{cases} 
        F''=kF\\
        \ddot{G}=c^2kG
    \end{cases}
\end{equation*}
\begin{align*}
    u(0,t)&=F(0)G(t)=u_0\mathrm{~}\forall t\geq0\quad\Rightarrow&\mathbf{F(0)}=\mathbf{u_0}\\
    u(L,t)&=F(L)G(t)=u_L\mathrm{~}\forall t\geq0\quad\Rightarrow&\mathbf{F(L)}=\mathbf{u_L}
\end{align*}
By applying the \textbf{IC} to $F''(x)=kF(x)$, $F(x)$ can be found:
\begin{align*}
    F(x)=&
    \begin{cases}
        A_1e^{\sqrt{k}x}+A_2e^{-\sqrt{k}x}&k>0\\
        A_1\cos(\sqrt{|k|}x)+A_2\sin(\sqrt{|k|}x)&k<0\\
        A_1x+A_2&k=0\\
    \end{cases}\\
\end{align*}
Since $k$ is constant for both $F,G$, $G(t)$ can be obtained by:
\begin{align*}
    G(t)=&
    \begin{cases}
        B_1e^{\sqrt{k}t}+B_2e^{-\sqrt{k}t}&\;\,k>0\\
        B_1\cos(\sqrt{|k|}t)+B_2\sin(\sqrt{|k|}t)&\;\,k<0\\
        B_1t+B_2&\;\,k=0
    \end{cases}
\end{align*}
%
%
\subsubsection{(iii) Specific Solution}
In general, the specific solution can be found by combining the boundary conditions \textbf{BC} 
with the general solution found in (ii).\\\hbox{}\\
If $\mathbf{u(0,t)=u(L,t)= 0}$ the \textbf{general solution} Ansatz:
\begin{equation*}
    u(x,t)=\sum\limits_{n=1}^{\infty}
    \Bigr( B_n \cos(\lambda _n t)+B_n^* \sin(\lambda_n t)\Bigl) \sin 
    \left(\frac{n \pi}{L}x \right)
\end{equation*}\\
can be used, and the \textbf{specific solution} by \textbf{FS} is given by:
\begin{align*}
    \lambda_n&=\frac{c\; n\; \pi}{L}\\
    B_n&=\frac{2}{L}\int\limits_0^L f(x)\sin(\frac{n\;\pi}{L}x)dx\\
    B_n^*&=\frac{2}{L\lambda_n} \int\limits_0^L g(x) \sin(\frac{n\;\pi}{L}x)dx\\\\
    f(x)&=\sum\limits_{n=1}^{\infty} B_n \sin(\frac{n\;\pi}{L}x)\\
    g(x)&=\sum\limits_{n=1}^{\infty} B_n^* \lambda_n \sin(\frac{n\;\pi}{L}x)\\
\end{align*}
\text{Useful tricks:}
\begin{itemize}
    \item If $f(x),g(x)$ can be written in terms of $\sin(\frac{n\;\pi}{L}x)$, $B_n, B_n^*$ 
    can be calculated directly
\end{itemize}
%
%
\subsubsection{Inhomogeneous Boundary Conditions}
To solve a linear homogeneous 2nd order PDE with inhomogeneous BC, a fitting substitution can be used:
\begin{align*}
    &v(x,t):= u(x,t) -w(x)\\
    &\begin{cases}
        v_{tt}=c^2v_{xx}-w_x\\
        v(0,t)=u(0,t)-w(0)=0\\
        v(L,t)=u(L,t)-w(L)=0\\
        v(x,0)=u(x,0)-w(x)\\
        v_t(x,0)=u_t(x,0)
    \end{cases}
\end{align*}
subsequently, $v(x,t)$ has homogeneous BC.\@
%
%
\begin{examplesection}[Example]
For the given Problem
\[ u_{tt}=c^2u_{xx}\]
we make the Ansatz
\[u(x,t)=F(x)G(t)\]
Plugging this into the given PDE and rearranging delivers
\[u_{tt}=F\ddot{G};\qquad u_{xx}=F''G \quad \rightarrow F\ddot{G}=c^2F''G\]
\[\frac{\ddot{G}}{c^2G}=\frac{F''}{F}=k\;\; ;\qquad\quad\begin{cases} F''=kF\\
\ddot{G}=c^2kG
\end{cases}\]
as $k$ must be constant. This rearrangement is the purpose of our Ansatz and delivers two ODEs to solve instead of one PDE.
Inserting the boundary conditions into $F''=kF$ and using a harmonic Ansatz delivers a set of solutions for the BVP of the form
\[F_n (x)=\sin{\left( \frac{n \pi}{L} x \right) }, \sqrt{-k}=\frac{n \pi}{L}\]
which satisfy the BVP for negative values of k ($k \ge 0$ yields the $F=0$ solution which is not interesting).\\
Inserting these values of $k$ into $\ddot{G}=c^2kG$ and using a harmonic Ansatz again delivers a family of solutions
\[G_n (t) =  B_n \cos{(\lambda_{n} t)} + B_n^* \sin{(\lambda_{n}t)}\]
where 
\[ \lambda_{n} = \frac{c n \pi}{L} \] 
Inserting $F_n$ and $G_n$ into $u(x,t)=F(x)G(t)$ we obtain a family of solutions for our initial problem of the form
\[u_n (x, t) = (B_n \cos{(\lambda_{n} t)} + B_n^* \sin{(\lambda_{n}t)})\sin{\left( \frac{n \pi}{L} x \right) } \]
where $u_n (x, t)$ are called Eigenfunctions and $\lambda_n$ the spectrum.\\
Using fourier series we can generate a general solution for all $n$ of the form
\[u(x,t)=\sum\limits_{n=1}^{\infty}\begin{pmatrix}
    B_n cos(\lambda _n t)+B_n^* sin(\lambda_n t)
    \end{pmatrix} sin\left(\frac{n \pi}{L}x \right) \]
which does not satisfy the IVP in genearal.\\
To satisfy the IVP we insert the initial value conditions and from $u(x,0)=f(x)$ we obtain \ref*{eq:formel3} and from $u'(x,0)=g(x)$ we obtain \ref*{eq:formel4}.
\end{examplesection}
\subsection{1D Wave Equation - d'Alambert}

