\section{Laplace Transform}
\subsection{Definition}
Let $f:\mathbb{R}_{\geq 0} \rightarrow \mathbb{R}$ be a function.
\begin{equation*}
    \boxed{
    f(t)\:\laplace\:F(s)=\mathcal{L}[f(t)](s):=\int_{0}^{\infty}e^{-st}f(t)\,dt
    }
\end{equation*}
Inverse Laplace transform:
\begin{equation*}
    F(s)\:\Laplace\:f(t)=\mathcal{L}^{-1}[F(s)](t)
\end{equation*}
\textbf{Remarks}
\begin{itemize}
    \item The Laplace Transform is a useful tool to solve ODE's with given initial conditions (IVP). See Section~\ref{ssec:laplaceDiff}.
    \item Two functions defined on $\mathbb{R}_{0^+}$ that have the same Laplace Transform can differ in single points but not on an interval.
\end{itemize}

\subsubsection{Growth Restriction}
If $f$ is piecewise continuous and there exist $M, k\in \mathbb{R}_{0^+}$ such that
\begin{equation*}
    \left|f(t)\right|\leq Me^{kt}
\end{equation*}
for all $t\geq 0, $ then $\mathcal{L}[f](s) $ exists on the interval $(k,\infty).$


\subsection{Linearity}
Let $f$ and $g$ be functions for which the Laplace transform exists. Then for all $a,b \in \mathbb{R}$
\begin{equation*}
    \mathcal{L}[a\cdot f(t)+b\cdot g(t)] = a\cdot\mathcal{L}[f(t)]+b\cdot\mathcal{L}[g(t)]
\end{equation*}
Moreover
\begin{equation*}
    \mathcal{L}^{-1}[a\cdot F(s)+b\cdot G(s)] = a\cdot\mathcal{L}^{-1}[F(s)]+b\cdot\mathcal{L}^{-1}[G(s)]
\end{equation*}

\begin{examplesection}[Example]
    Let $f(t)=2t+e^t$, then
    \begin{equation*}
        F(s)=\mathcal{L}[f(t)]=\mathcal{L}[2t+e^t]=2\cdot\mathcal{L}[t] + \mathcal{L}[e^t]=\frac{2}{s^2}+\frac{1}{s-1}
    \end{equation*}
    \hrule{}
    Let $F(s)=\frac{4}{s^5}$, then
    \begin{equation*}
        f(t)=\mathcal{L}^{-1}[F(s)]=\mathcal{L}^{-1}\left[\frac{24}{s^5}\cdot\frac16\right]=\frac16\mathcal{L}^{-1}\left[\frac{24}{s^5}\right]=\frac16t^4
    \end{equation*}
    \hrule{}
    Let $F(s)=\frac{a}{bs+c}; a,b,c \in \mathbb{R}$, then
    \begin{equation*}
        F(s)=\frac{a}{bs+c}=\frac{\frac{a}{b}}{s+c/b}=\frac{\frac{a}{b}}{s-(-\frac{c}{b})}
    \end{equation*}
    \begin{equation*}
        f(t)=\mathcal{L}^{-1}[F(s)]=\frac{a}{b}\mathcal{L}^{-1}\left[\frac{1}{s-(-\frac{c}{b})}\right]=\frac{a}{b}e^{-\frac{c}{b}t}
    \end{equation*}
\end{examplesection}

\subsection{s-shifting}
\begin{equation*}
    e^{at}\cdot f(t)\:\laplace\:F(s-a)
\end{equation*}
\begin{examplesection}[Example]
    Let $f(t)=e^{2t}\cdot \cos(\omega t)$ $\Rightarrow a=2$ (s-shifting)
    \begin{equation*}
        \mathcal{L}[\cos(\omega t)]=\frac{s}{s^2+\omega ^2}\Rightarrow \mathcal{L}[f(t)]=\frac{s-2}{{(s-2)}^2+\omega ^2}
    \end{equation*}
\end{examplesection}

\subsection{Differentiation}\label{ssec:laplaceDiff}
Let $f, f', \dots, f^{(n-1)}$ be continuous functions for all $t \geq 0$. Assume that $f^{(n)}$ is piecewise continuous on every finite subinterval in $[0;1)$. Then % chktex 9
\begin{equation*}
    \mathcal{L}[f^{(n)}](s)=s^n\mathcal{L}[f](s)-\sum_{j=0}^{n-1}s^{n-1-j}f^{(j)}(0),\text{ for every }n\geq1
\end{equation*}
For n = 1, n=2 and n=3
\begin{align*}
    \mathcal{L}[f'](s)   & =s\mathcal{L}[f](s)-f(0)                    \\
    \mathcal{L}[f''](s)  & =s^2\mathcal{L}[f](s)-sf(0)-f'(0)           \\
    \mathcal{L}[f'''](s) & =s^3\mathcal{L}[f](s)-s^2f(0)-sf'(0)-f''(0) % chktex 23
\end{align*}
\begin{examplesection}[Example]
    Give the ODE $y''-y'+y=0,\;y(0)=0,\;y'(0)=1$
    \begin{align*}
        \mathcal{L}[y''-y'+y]                                                               & \:=\:\mathcal{L}[0] \Leftrightarrow                                                     \\
        \mathcal{L}[y'']-\mathcal{L}[y']+\mathcal{L}[y]                                     & \:=\:\mathcal{L}[0] \Leftrightarrow                                                     \\
        s^2Y -\underbrace{sy(0)}_{=0}-\underbrace{y'(0)}_{=1}-(sY-\underbrace{y(0)}_{=0})+Y & \:=\:0 \Leftrightarrow                                                                  \\
        Y=\frac{1}{s^2-s+1}=\frac{1}{{(s-\frac{1}{2})}^2+\frac{3}{4}}=\sqrt{\frac{4}{3}}    & \frac{\sqrt{\frac{3}{4}}}{{(s-\frac{1}{2})}^2+{(\sqrt{\frac{3}{4}})}^2} \Leftrightarrow \\
        Y\;\Laplace\;y=\sqrt{\frac{4}{3}}e^{\frac{1}{2}t}\sin\left(\sqrt{\frac{3}{4}t}\right)
    \end{align*}
\end{examplesection}

\subsection{Differentiation of transforms}
Let $f$ be a piecewise continuous function that satisfies the growth condition. Then
\begin{equation*}
    \mathcal{L}'[f(t)] = \frac{d}{ds}\mathcal{L}[f(t)] = -\mathcal{L}[t f(t)]
\end{equation*}
\begin{equation*}
    \mathcal{L}^{-1}[F'(s)] = -tf(t)
\end{equation*}

\subsection{Heaviside function}
\begin{tabular}{ m{5cm}  m{3.4cm} }
    \begin{equation*}
        u(t):=
        \begin{cases}
            1 & \text{if}\;t>0  \\
            0 & \text{if}\;t<0
        \end{cases}
    \end{equation*}
     & 
    \begin{tikzpicture}
        \begin{axis}[
                width=\linewidth,
                unit vector ratio={1 1},
                axis x line=left,
                axis y line=middle,
                xmin=-1.5,
                xmax=1.5,
                ymin=0,
                ymax=2,
                xlabel={$t$},
                ylabel={$u(t)$},
                xtick={0},
                ytick={1},
                mark=none,
            ]
            \addplot [blue, very thick]
            coordinates {
                    (\pgfkeysvalueof{/pgfplots/xmin},0)
                    (0,0)
                };
            \addplot [blue, very thick]
            coordinates {
                    (0,1)
                    (\pgfkeysvalueof{/pgfplots/xmax},1)
                };
            \addplot[fill=white,only marks,mark=*] coordinates{(0,0)(0,1)};
        \end{axis}
    \end{tikzpicture}
    \\
    \begin{equation*}
        u(t-a):=
        \begin{cases}
            1 & \text{if}\;t>a  \\
            0 & \text{if}\;t<a
        \end{cases}
    \end{equation*}
     & 
    \begin{tikzpicture}
        \begin{axis}[
                width=\linewidth,
                unit vector ratio={1 1},
                axis x line=left,
                axis y line=middle,
                xmin=-0.5,
                xmax=2.5,
                ymin=0,
                ymax=2,
                xlabel={$t$},
                ylabel={$u(t-a)$},
                xtick={0},
                ytick={1},
                extra x ticks={1},
                extra x tick labels={$a$},
                mark=none,
            ]
            \addplot [blue, very thick]
            coordinates {
                    (\pgfkeysvalueof{/pgfplots/xmin},0)
                    (1,0)
                };
            \addplot [blue, very thick]
            coordinates {
                    (1,1)
                    (\pgfkeysvalueof{/pgfplots/xmax},1)
                };
            \addplot[fill=white,only marks,mark=*] coordinates{(1,0)(1,1)};
        \end{axis}
    \end{tikzpicture}
    \\
    \begin{equation*}
        f(t):=
        \begin{cases}
            1 & a<t<b       \\
            0 & \text{else}
        \end{cases}
    \end{equation*}
    \begin{equation*}
        f(t)=u(t-a)-u(t-b)
    \end{equation*}
     & 
    \begin{tikzpicture}
        \begin{axis}[
                width=\linewidth,
                unit vector ratio={1 1},
                axis x line=left,
                axis y line=middle,
                xmin=-0.5,
                xmax=2.5,
                ymin=0,
                ymax=2,
                xlabel={$t$},
                ylabel={$f(t)$},
                xtick={0},
                ytick={1},
                extra x ticks={0.7, 1.8},
                extra x tick labels={$a$, $b$},
                mark=none,
            ]
            \addplot [blue, very thick]
            coordinates {
                    (\pgfkeysvalueof{/pgfplots/xmin},0)
                    (0.7,0)
                };
            \addplot [blue, very thick]
            coordinates {
                    (0.7,1)
                    (1.8,1)
                };
            \addplot [blue, very thick]
            coordinates {
                    (1.8,0)
                    (\pgfkeysvalueof{/pgfplots/xmax},0)
                };
            \addplot[fill=white,only marks,mark=*] coordinates{(0.7,0)(0.7,1)(1.8,1)(1.8,0)};
        \end{axis}
    \end{tikzpicture}
    \\
\end{tabular}

\subsection{t-shifting}
\begin{align*}
    f(t-a)u(t-a) & \:\laplace\: e^{-as}F(s)                \\
    f(t)u(t-a)   & \:\laplace\: e^{-as}\mathcal{L}[f(t+a)] \\
    f(t)u(t)     & \:\laplace\: \mathcal{L}[f]
\end{align*}
\begin{examplesection}[Example]
    \setlength{\abovedisplayskip}{0pt}
    \setlength{\belowdisplayskip}{0pt}
    Let $f(t)=t^2$ and now shift it by $a$: $f(t-a)={(t-a)}^2$
    \begin{equation*}
        \mathcal{L}[f]=e^{-as}\frac{2}{s^3}
    \end{equation*}
    \hrule{}
    Given
    \begin{align*}
        F(s)                & =\frac{e^{-2s}}{s^4}=\frac{1}{6}e^{-2s}\underbrace{\frac{3\!}{s^4}}_{=\mathcal{L}[t^3]} \\
        \mathcal{L}^{-1}[F] & = \frac{1}{6}{(t-2)}^3u(t-2)=\begin{cases}\frac{1}{6}{(t-2)}^3& t>2\\ 0& t<2\end{cases}
    \end{align*}
\end{examplesection}

\subsection{Integration}
Let $f$ be a piecewise continuous function for $t\geq0$
\begin{equation*}
    \int_{0}^{t}f(x)dx \:\laplace\: \frac{1}{s}F(s)
\end{equation*}

\subsection{Integration of transforms}
If the $\lim_{t\to 0^+}\frac{f(t)}{t}$ exists, then
\begin{equation*}
    \int_{s}^{\infty}\mathcal{L}[f](s^{\prime})\mathrm{~}ds^{\prime}=\mathcal{L}\left[\frac{f(t)}t\right](s)
\end{equation*}

\subsection{Dirac's delta function}
The \textit{Dirac delta} is the limit
\begin{equation*}
    \delta(t-a):=\lim_{h\to0}\delta_h(t-a)\:\text{''}=\text{''}
    \begin{cases}
        \infty & t=a     \\
        0      & t\ne a
    \end{cases}
\end{equation*}
and
\begin{equation*}
    \int_{-\infty}^{\infty}\delta(t-a)dt=1
\end{equation*}
Let $a \in \mathbb{R}, a \ne 0$. Then
\begin{equation*}
    \delta(at)=\frac1{|a|}\delta(t)
\end{equation*}
\begin{equation*}
    \int_{0}^{\infty} g(t)\delta(t-a)dt=g(a)
\end{equation*}
\begin{equation*}
    u(t)=\int_{-\infty}^{t}\delta(r)dr
\end{equation*}
Let $f(t)$ be a differentiable function with continuous derivative and with zeros $t_1,\dots,t_n$. Then
\begin{equation*}
    \delta(f(t))=\sum_{j=1}^n\frac{\delta(t-t_j)}{|f'(t_j)|}
\end{equation*}
The Laplace transform is given by
\begin{align*}
    \delta(t)   & \;\laplace\; 1       \\
    \delta(t-a) & \;\laplace\; e^{-as}
\end{align*}

\subsection{Convolution}
\textbf{General Form}\\
The convolution
\begin{equation*}
    (f*g)(t):=\int_{-\infty}^{\infty}f(\tau)g(t-\tau)d\tau
\end{equation*}
of $f$ and $g$ exists if $f,g\in L^1$ i. e.
\begin{equation*}
    \int_{-\infty}^{\infty}\left|f(x)\right|dx<\infty\quad\text{ and }\quad\int_{-\infty}^{\infty}\left|g(x)\right|dx<\infty
\end{equation*}

\textbf{Functions on $\mathbb{R}_{0^+}$}\\
If $f,g=0$ for $t<0$ the convolution integral becomes
\begin{equation*}
    (f*g)(t):=\int_{0}^{t}f(\tau)g(t-\tau)d\tau
\end{equation*}
\textbf{Properties}: Let $f,g$ and $h$ be functions.
\begin{align*}
    f*g =     & \ g*f                               \\
    f*(g+h) = & \ f*g+f*h                           \\
    f*(g*h) = & \ (f*g)*h                           \\
    f*0 =     & \ 0*f=0                             \\
    f*1 \ne   & \ f                                 \\
    f*f \;    & \ \text{is not always non-negative}
\end{align*}
\begin{equation*}
    f(t)*g(t) \:\laplace\: F(s)\cdot G(s)
\end{equation*}

\subsection{Solving initial value problems}
\begin{enumerate}
    \item Transform the ODE to the s-domain ($\mathcal{L}[y]=Y$)\\
          $\Rightarrow$ Insert initial conditions\\
          $\quad\qquad y''+ay'+by=r(t)$\\
          $\quad\qquad(s^2Y-sy(0)-y'(0))+a(sY-y(0))+bY=R(s)$
    \item Solve the equation for $Y$ \\
          $\quad\qquad (s^2+as+b)Y=R(s)+sy(0)+y'(0)+ay(0)$
    \item Transform the equation back to the time-domain $y=\mathcal{L}^{-1}[Y]$
\end{enumerate}
Special case:\\
If the initial values are given like this $y(a),y'(a),\dots$:
\begin{enumerate}[label=\alph*]
    \item Substitute: $t=\tilde{t}+a$
    \item $y''+ay'+by=r(t) \Rightarrow \tilde{y}''+a\tilde{y}'+b\tilde{y}=r(\tilde{t}+a)$\\
          $\tilde{y}(0)=y(a),\:\tilde{y}'(0)=y'(a),\dots$
    \item Solve the ODE as described above $\Rightarrow \tilde{Y} \rightarrow \tilde{y}(\tilde{t})$
    \item Resubstitute: $\tilde{t}=t-a$; $\tilde{y}(\tilde{t})\rightarrow y(t)$
\end{enumerate}