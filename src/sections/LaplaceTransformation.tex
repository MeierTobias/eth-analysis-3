\section{Laplace Transform}
\subsection{Definition}
Let $f:\mathbb{R}_{\geq 0} \rightarrow \mathbb{R}$ be a function.
\begin{equation*}
    \boxed{
    f(t)\:\laplace\:F(s)=\mathcal{L}[f(t)](s):=\int_{0}^{\infty}e^{-st}f(t)\,dt 
    }
\end{equation*}
Inverse Laplace transform:
\begin{equation*}
    F(s)\:\Laplace\:f(t)=\mathcal{L}^{-1}[F(s)]
\end{equation*} 

\subsection{Linearity}
Let $f$ and $g$ be functions for which the Laplace transform exists. Then for all $a,b \in \mathbb{R}$
\begin{equation*}
    \mathcal{L}[a\cdot f(t)+b\cdot g(t)] = a\cdot\mathcal{L}[f(t)]+b\cdot\mathcal{L}[g(t)]
\end{equation*}
Moreover
\begin{equation*}
    \mathcal{L}^{-1}[a\cdot F(s)+b\cdot G(s)] = a\cdot\mathcal{L}^{-1}[F(s)]+b\cdot\mathcal{L}^{-1}[G(s)]
\end{equation*}

\begin{examplesection}[Example]
    Let $f(t)=2t+e^t$, then
    \begin{equation*}
        F(s)=\mathcal{L}[f(t)]=\mathcal{L}[2t+e^t]=2\cdot\mathcal{L}[t] + \mathcal{L}[e^t]=\frac{2}{s^2}+\frac{1}{s-1}
    \end{equation*}
    \hrule{}
    Let $F(s)=\frac{4}{s^5}$, then
    \begin{equation*}
        f(t)=\mathcal{L}^{-1}[F(s)]=\mathcal{L}^{-1}\left[\frac{24}{s^5}\cdot\frac16\right]=\frac16\mathcal{L}^{-1}\left[\frac{24}{s^5}\right]=\frac16t^4
    \end{equation*}
    \hrule{}
    Let $F(s)=\frac{a}{bs+c}; a,b,c \in \mathbb{R}$, then
    \begin{equation*}
        F(s)=\frac{a}{bs+c}=\frac{\frac{a}{b}}{s+c/b}=\frac{\frac{a}{b}}{s-(-\frac{c}{b})}
    \end{equation*}
    \begin{equation*}
        f(t)=\mathcal{L}^{-1}[F(s)]=\frac{a}{b}\mathcal{L}^{-1}\left[\frac{1}{s-(-\frac{c}{b})}\right]=\frac{a}{b}e^{-\frac{c}{b}t}
    \end{equation*}
\end{examplesection}

\subsection{s-shifting}
\begin{equation*}
    e^{at}\cdot f(t)\:\laplace\:F(s-a)
\end{equation*}

\subsection{Differentiation}
Let $f, f', \dots, f^{(n-1)}$ be continuous functions for all $t \geq 0$. Assume that $f^{(n)}$ is piecewise continuous on every finite subinterval in $[0;1)$. Then
\begin{equation*}
    \mathcal{L}[f^{(n)}](s)=s^n\mathcal{L}[f](s)-\sum_{j=0}^{n-1}s^{n-1-j}f^{(j)}(0),\text{ for every }n\geq1
\end{equation*}
For n = 1, n=2 and n=3
\begin{align*}
    \mathcal{L}[f'](s)&=s\mathcal{L}[f](s)-f(0) \\
    \mathcal{L}[f''](s)&=s^2\mathcal{L}[f](s)-sf(0)-f'(0) \\
    \mathcal{L}[f'''](s)&=s^3\mathcal{L}[f](s)-s^2f(0)-sf'(0)-f''(0)
\end{align*}

\subsection{Differentiation of transforms}
Let $f$ be a piecewise continuous function that satisfies the growth condition. Then
\begin{equation*}
    \mathcal{L}'[f(t)] = \frac{d}{ds}\mathcal{L}[f(t)] = -\mathcal{L}[t f(t)]
\end{equation*}
\begin{equation*}
    \mathcal{L}^{-1}[F'(s)] = -tf(t)
\end{equation*}

\subsection{Heaviside function}
\begin{tabular}{ m{5cm}  m{3.4cm} }
    \begin{equation*}
        u(t):=
        \begin{cases}
            1&\text{if}\;t>0 \\
            0&\text{if}\;t<0
        \end{cases}
    \end{equation*}
    &  
    \begin{tikzpicture}
        \begin{axis}[
            width=\linewidth,
            unit vector ratio={1 1},
            axis x line=left,
            axis y line=middle,
            xmin=-1.5,
            xmax=1.5,
            ymin=0,
            ymax=2,
            xlabel={$t$},
            ylabel={$u(t)$},
            xtick={0},
            ytick={1},
            mark=none,
        ]
            \addplot [blue, very thick] 
            coordinates {
                (\pgfkeysvalueof{/pgfplots/xmin},0)
                (0,0)
            };
            \addplot [blue, very thick] 
            coordinates {
                (0,1)
                (\pgfkeysvalueof{/pgfplots/xmax},1)
            };
            \addplot[fill=white,only marks,mark=*] coordinates{(0,0)(0,1)};
        \end{axis}
    \end{tikzpicture}
    \\
    \begin{equation*}
        u(t-a):=
        \begin{cases}
            1&\text{if}\;t>a \\
            0&\text{if}\;t<a
        \end{cases}
    \end{equation*}
    &
    \begin{tikzpicture}
        \begin{axis}[
            width=\linewidth,
            unit vector ratio={1 1},
            axis x line=left,
            axis y line=middle,
            xmin=-0.5,
            xmax=2.5,
            ymin=0,
            ymax=2,
            xlabel={$t$},
            ylabel={$u(t-a)$},
            xtick={0},
            ytick={1},
            extra x ticks={1},
            extra x tick labels={$a$},
            mark=none,
        ]
            \addplot [blue, very thick] 
            coordinates {
                (\pgfkeysvalueof{/pgfplots/xmin},0)
                (1,0)
            };
            \addplot [blue, very thick] 
            coordinates {
                (1,1)
                (\pgfkeysvalueof{/pgfplots/xmax},1)
            };
            \addplot[fill=white,only marks,mark=*] coordinates{(1,0)(1,1)};
        \end{axis}
    \end{tikzpicture}
    \\
    \begin{equation*}
        f(t):=
        \begin{cases}
            1&a<t<b \\
            0&\text{else}
        \end{cases}
    \end{equation*}
    \begin{equation*}
        f(t)=u(t-a)-u(t-b)
    \end{equation*}
    &
    \begin{tikzpicture}
        \begin{axis}[
            width=\linewidth,
            unit vector ratio={1 1},
            axis x line=left,
            axis y line=middle,
            xmin=-0.5,
            xmax=2.5,
            ymin=0,
            ymax=2,
            xlabel={$t$},
            ylabel={$f(t)$},
            xtick={0},
            ytick={1},
            extra x ticks={0.7, 1.8},
            extra x tick labels={$a$, $b$},
            mark=none,
        ]
            \addplot [blue, very thick] 
            coordinates {
                (\pgfkeysvalueof{/pgfplots/xmin},0)
                (0.7,0)
            };
            \addplot [blue, very thick] 
            coordinates {
                (0.7,1)
                (1.8,1)
            };
            \addplot [blue, very thick] 
            coordinates {
                (1.8,0)
                (\pgfkeysvalueof{/pgfplots/xmax},0)
            };
            \addplot[fill=white,only marks,mark=*] coordinates{(0.7,0)(0.7,1)(1.8,1)(1.8,0)};
        \end{axis}
    \end{tikzpicture}
    \\
\end{tabular}

\subsection{t-shifting}
\begin{align*}
    f(t-a)u(t-a) &\:\laplace\: e^{-as}F(s)\\
    f(t)u(t-a) &\:\laplace\: e^{-as}Ff(t+a)
\end{align*}

\subsection{Integration}
Let $f$ be a piecewise continuous function for $t\geq0$
\begin{equation*}
    \int_{0}^{t}f(x)dx \:\laplace\: \frac{1}{s}F(s)
\end{equation*}

\subsection{Integration of transforms}
If the $\lim_{t\to 0^+}\frac{f(t)}{t}$ exists, then
\begin{equation*}
    \int_{s}^{\infty}\mathcal{L}[f](s^{\prime})\mathrm{~}ds^{\prime}=\mathcal{L}\left[\frac{f(t)}t\right](s)
\end{equation*}

\subsection{Dirac's delta function}
The \textit{Dirac delta} is the limit
\begin{equation*}
    \delta(t-a):=\lim_{h\to0}\delta_h(t-a)"="
    \begin{cases}
        \infty&t=a\\
        0&t\ne a
    \end{cases}
\end{equation*}
and
\begin{equation*}
    \int_{-\infty}^{\infty}\delta(t-a)dt=1
\end{equation*}
Let $a \in \mathbb{R}, a \ne 0$. Then
\begin{equation*}
    \delta(at)=\frac1{|a|}\delta(t)
\end{equation*}
\begin{equation*}
    \int_{0}^{\infty} g(t)\delta(t-a)dt=g(a)
\end{equation*}
\begin{equation*}
    u(t)=\int_{-\infty}^{t}\delta(r)dr
\end{equation*}
Let $f(t)$ be a differentiable function with continuous derivative and with zeros $t_1,\dots,t_n$. Then
\begin{equation*}
    \delta(f(t))=\sum_{j=1}^n\frac{\delta(t-t_j)}{|f'(t_j)|}
\end{equation*}
The Laplace transform is given by
\begin{align*}
    \delta(t) &\;\laplace\; 1 \\
    \delta(t-a) &\;\laplace\; e^{-as}
\end{align*}

\subsection{Convolution}
\begin{equation*}
    f*g(t):=\int_{0}^{t}f(\tau)g(t-\tau)d\tau
\end{equation*}
Properties: Let $f,g$ and $h$ be functions.
\begin{align*}
    f*g =&\ g*f \\
    f*(g+h) =&\ f*g+f*h \\
    f*(g*h) =&\ (f*g)*h \\
    f*0 =&\ 0*f=0 \\
    f*1 \ne&\ f \\
    f*f \;&\ \text{is not always non-negative}
\end{align*}
\begin{equation*}
    f(t)*g(t) \:\laplace\: F(s)\cdot G(s)
\end{equation*}

\subsection{Solving initial value problems}

\subsection{Laplace transform table}
\begin{align*}
    f(t)                            \;&\laplace\; F(s) \\
    1                               \;&\laplace\; \frac{1}{s} \\
    t^{n}                           \;&\laplace\; \frac{n!}{s^{n+1}} \; with \; n \in \mathbb{N}_0 \\
    e^{at}                          \;&\laplace\; \frac{1}{s-a}  \\
    e^{at}f(t)                      \;&\laplace\; F(s-a) \\
    u(t-a)                          \;&\laplace\; \frac{1}{s}e^{-as} \\
    f(t-a)u(t-a)                    \;&\laplace\; e^{-as}F(s) \\
    \delta(t)                       \;&\laplace\; 1\\
    \delta(t-t_{0})                 \;&\laplace\; e^{-st_{0}} \\
    \frac{d^{n}}{dt^{n}}\delta(t)   \;&\laplace\; s^{n}  \\
    t^{n}f(t)                       \;&\laplace\; (-1)^n \frac{d^nF(s)}{ds^n}  \\
    f^{\prime}\left(t\right)        \;&\laplace\; \begin{aligned}sF(s)-f(0)\end{aligned}  \\
    f^{n}(t)                        \; \laplace\; s^nF(s)&-s^{n-1}f(0)-...-f^{n-1}(0)  \\
    f(t)*g(t)                       \;&\laplace\; \begin{aligned}F(s)\cdot G(s)\end{aligned}  \\
    ln(at)                          \;&\laplace\; -\frac{1}{s}\left(ln\left(\frac sa\right)+\gamma\right)  \\ 
    \frac{e^{at}-e^{bt}}{a-b}       \;&\laplace\; \frac{1}{(s-a)(s-b)} \\
    \frac{ae^{at}-be^{at}}{a-b}     \;&\laplace\; \frac{s}{(s-a)(s-b)} \\
    te^{at}                         \;&\laplace\; \frac{1}{(s-a)^2}  \\
    t^{n}e^{at}                     \;&\laplace\; \frac{n!}{(s-a)^{n+1}}  \\
    -tf(t)                          \;&\laplace\; F^{\prime}(s)  \\
    t^{2}f(t)                       \;&\laplace\; F^{\prime\prime}(s)  \\
    (-t)^n f(t)                     \;&\laplace\; F^{(n)}(s)  \\
    \int_{0}^{t}f(u)du              \;&\laplace\; \frac{1}{s}F(s)  \\
    \int_{0}^{t}\frac{(t-q)^{n-1}f(q)}{(n-1)!}dq,&n \leq 1 \;\laplace\; \frac{1}{s^n}F(s),n \geq 1  \\
    \frac{1}{t}f(t)                 \;&\laplace\; \int_{s}^{\infty}F(u)du  \\
    1-e^{-at}                       \;&\laplace\; \frac{a}{s(s+a)}  \\
    sin(kt)                         \;&\laplace\; \frac{k}{s^2+k^2}  \\
    cos(kt)                         \;&\laplace\; \frac{s}{s^2+k^2}  \\
    sin^{2}(kt)                     \;&\laplace\; \frac{2k^2}{s(s^2+4k^2)}  \\
    cos^{2}(kt)                     \;&\laplace\; \frac{s^2+2k^2}{s(s^2+4k^2)}  \\
    sinh(kt)                        \;&\laplace\; \frac{k}{s^2-k^2} \\
    cosh(kt)                        \;&\laplace\; \frac{s}{s^2-k^2} \\
    e^{at}sin(kt)                   \;&\laplace\; \frac{k}{(s-a)^2+k^2}  \\
    e^{at}cos(kt)                   \;&\laplace\; \frac{s-a}{(s-a)^2+k^2}  \\
    e^{at}sinh(kt)                  \;&\laplace\; \frac{k}{(s-a)^2-k^2}  \\
    e^{at}cosh(kt)                  \;&\laplace\; \frac{(s-a)}{(s-a)^2-k^2}  \\
    t\cdot sin(kt)                        \;&\laplace\; \frac{2ks}{(s^2+k^2)^2}  \\
    t\cdot cos(kt)                        \;&\laplace\; \frac{s^2-k^2}{(s^2+k^2)^2}  \\
    t\cdot sin(t)cos(t)                   \;&\laplace\; \frac{2s}{(s^2+4)^2}  \\
    t\cdot sinh(kt)                       \;&\laplace\; \frac{2ks}{(s^2-k^2)^2}  \\
    t\cdot cosh(kt)                       \;&\laplace\; \frac{s^2-k^2}{(s^2-k^2)^2}  \\
    sin(at)\cdot f(t)               \;&\laplace\; \frac{1}{2i}\cdot(F(s-ia)-F(s+ia)) \\
    cos(at)\cdot f(t)               \;&\laplace\; \frac{1}{2}\cdot(F(s-ia)+F(s+ia)) \\
    sinh(at)\cdot f(t)              \;&\laplace\; \frac{1}{2}\cdot(F(s-a)-F(s+a)) \\
    cosh(at)\cdot f(t)              \;&\laplace\; \frac{1}{2}\cdot(F(s-a)+F(s+a)) \\
    \frac{1}{\sqrt{\pi t}e^{\frac{-a^2}{4t}}} \;&\laplace\; \frac{e^{-a\sqrt{s}}}{\sqrt{s}} \\
    \frac{a}{2\sqrt{\pi t^3}}e^{\frac{-a^2}{4t}} \;&\laplace\; e^{-a\sqrt{s}}
\end{align*}
