\section{Fourier Analysis}

\subsection{Periodicity}
A function $f(x)$ is called periodic if it is defined for “most” $x \in \mathbb{R} $ and if there is a positive real number $p \in R, p > 0$ such that $f(x) = f(x+p)$ for all $x$.
\begin{align*}
    f(t+p)=f(t),\quad &p>0\\
    \sin\biggl( \underbrace{\frac{2\pi}{L}}_{\frac{2\pi}{p}}nt \biggr) &\rightarrow p=L \\
\end{align*}

\subsection{Trigonometric Properties}
\begin{align*}
    \int_{-L}^L\cos\left(\frac{n\pi}Lx\right)\cos\left(\frac{m\pi}Lx\right)dx=\begin{cases}0&n\neq m\\L&n=m\neq0\\2&Ln=m=0.\end{cases}\\
    \int_{-L}^L\sin\left(\frac{n\pi}Lx\right)\sin\left(\frac{m\pi}Lx\right)dx=\begin{cases}0&n\neq m\\L&n=m\neq0.\end{cases}\\
    \int_{-L}^L\cos\left(\frac{n\pi}Lx\right)\sin\left(\frac{m\pi}Lx\right)dx=0,\text{ for every }n,m.
\end{align*}

\subsection{Fourier Series}
For the Fourier series to exist and for it to converge to $f(t)$, $f(t)$ has to be defined and $2L$-periodic on $[-L,L]$. 
For every discontinouity, the \textbf{Dirichlet Theorem} has to hold. $a_0,a_n,b_b\in \mathbb{R}$
\begin{align*}
    f(x)&=a_0+\sum_{n=1}^\infty\left(a_n\cos\left(\frac{n\pi}Lx\right)+b_n\sin\left(\frac{n\pi}Lx\right)\right) \\ \\
    a_{0} &=\frac{1}{2L}\int_{-L}^{L}f(x)dx  \\ \\
    a_{m} &=\frac{1}{L}\int_{-L}^{L}f(x)\cos\left(\frac{m\pi}{L}x\right)dx,\text{ if }m>0 \\ \\
    b_{m}&=\frac{1}{L}\int_{-L}^{L}f(x)\sin\left(\frac{m\pi}{L}x\right)dx,\text{ if }m>0
\end{align*}
\subsubsection{Dirichlet Theorem - Convergenve}
For discontinouities $f(x^-)\neq f(x^+)$ the fourier series converges to:
\begin{align*}
    F(x_0)&=\frac12(f(x_0^-)+f(x_0^+))
\end{align*}

\subsubsection{Even FS}
Functions are even if $f(x)=f(-x)$ and due to symmetry:
\begin{align*}
    \int_{-L}^Lg(x)dx=2\int_0^Lg(x)dx
\end{align*}
\textbf{Fourier series} for even functions:
\begin{align*}
    f(x)&=a_0+\sum_{n=1}^{\infty}a_n\cos\left(\frac{n\pi}{L}x\right) \\ \\
    a_{0} &=\frac{1}{L}\int_{0}^{L}f(x)dx  \\
    a_{n} &=\frac2L\int_0^Lf(x)\cos\left(\frac{n\pi}Lx\right)dx \\
    b_{n} &=0
\end{align*}
\begin{align*}
    f(a\cdot x)=\frac1a\intop_0^\infty A(\frac\omega a)\cdot cos(\omega x)d\omega 
\end{align*}
\subsubsection{Odd FS}
Functions are odd if $f(x)=-f(-x)$ and due to symmetry:
\begin{align*}
    \int_{-L}^Lg(x)dx=0
\end{align*}
\textbf{Fourier series} for odd functions:
\begin{align*}
    f(x)&=\sum_{n=1}^\infty b_n\sin\left(\frac{n\pi}Lx\right) \\ \\
    a_0 &= a_n = 0\\
    b_n&=\dfrac{2}{L}\int_0^Lf(x)\sin\left(\dfrac{n\pi}{L}x\right)dx
\end{align*}

\subsubsection{Expansion of FS}
If $f(t)$ is defined on $[0,L]$:\vspace*{8pt}\\ 
\begin{minipage}[t]{0.18\textwidth}
    \textbf{Copy-paste} exp.\\   
    \begin{tikzpicture}
        \begin{axis}[
            width=\linewidth,
            unit vector ratio={3 1.2},
            axis x line=left,
            axis y line=middle,
            xmin=-2.5,
            xmax=2.5,
            ymin=0,
            ymax=2,
            %xlabel={$t$},
            %ylabel={$u(t-a)$},
            xtick={0},
            ytick={1},
            extra x ticks={-2, 2},
            extra x tick labels={$-L$,$L$},
            mark=none,
        ]
            \addplot [blue, very thick] 
            coordinates {
                %(\pgfkeysvalueof{/pgfplots/xmin},0)
                (0,1)
                (1,1)
            };
            \addplot [blue, very thick] 
            coordinates {
                (1,0.05)
                (2,0.05)
                %(\pgfkeysvalueof{/pgfplots/xmax},1)
            };
            \addplot [blue, very thick, dotted] 
            coordinates {
                %(\pgfkeysvalueof{/pgfplots/xmin},0)
                (-2,1)
                (-1,1)
            };
            \addplot [blue, very thick, dotted] 
            coordinates {
                (-1,0.05)
                (0,0.05)
                %(\pgfkeysvalueof{/pgfplots/xmax},1)
            };
            \addplot[fill=white,only marks,mark=*] coordinates{(1,0)(1,1)(0,1)(2,0)(-2,1)(-1,1)(-1,0)(0,0)};
        \end{axis}
    \end{tikzpicture}
\end{minipage}
\begin{minipage}[t]{0.18\textwidth}
    \textbf{Even} expansion\\
    \begin{tikzpicture}
        \begin{axis}[
            width=\linewidth,
            unit vector ratio={3 1.2},
            axis x line=left,
            axis y line=middle,
            xmin=-2.5,
            xmax=2.5,
            ymin=0,
            ymax=2,
            %xlabel={$t$},
            %ylabel={$u(t-a)$},
            xtick={0},
            ytick={1},
            extra x ticks={-2, 2},
            extra x tick labels={$-L$,$L$},
            mark=none,
        ]
            \addplot [blue, very thick] 
            coordinates {
                %(\pgfkeysvalueof{/pgfplots/xmin},0)
                (0,1)
                (1,1)
            };
            \addplot [blue, very thick] 
            coordinates {
                (1,0.05)
                (2,0.05)
                %(\pgfkeysvalueof{/pgfplots/xmax},1)
            };
            \addplot [blue, very thick, dotted] 
            coordinates {
                %(\pgfkeysvalueof{/pgfplots/xmin},0)
                (-1,1)
                (0,1)
            };
            \addplot [blue, very thick, dotted] 
            coordinates {
                (-2,0.05)
                (-1,0.05)
                %(\pgfkeysvalueof{/pgfplots/xmax},1)
            };
            \addplot[fill=white,only marks,mark=*] coordinates{(1,0)(1,1)(0,1)(2,0)(-2,0)(-1,1)(-1,0)(0,0)};
        \end{axis}
    \end{tikzpicture}
\end{minipage}
\begin{minipage}[t]{0.18\textwidth}
    \textbf{Odd} expansion\\
    \begin{tikzpicture}
        \begin{axis}[
            width=\linewidth,
            unit vector ratio={3 1.2},
            axis x line=center,
            axis y line=center,
            xmin=-2.5,
            xmax=2.5,
            ymin=-2,
            ymax=2,
            %xlabel={$t$},
            %ylabel={$u(t-a)$},
            xtick={0},
            ytick={1},
            extra x ticks={-2,2},
            extra x tick labels={$-L$,$L$},
            mark=none,
        ]
            \addplot [blue, very thick] 
            coordinates {
                %(\pgfkeysvalueof{/pgfplots/xmin},0)
                (0,1)
                (1,1)
            };
            \addplot [blue, very thick] 
            coordinates {
                (1,0.05)
                (2,0.05)
                %(\pgfkeysvalueof{/pgfplots/xmax},1)
            };
            \addplot [blue, very thick, dotted] 
            coordinates {
                %(\pgfkeysvalueof{/pgfplots/xmin},0)
                (-1,-1)
                (0,-1)
            };
            \addplot [blue, very thick, dotted] 
            coordinates {
                (-2,-0.05)
                (-1,-0.05)
                %(\pgfkeysvalueof{/pgfplots/xmax},1)
            };
            \addplot[fill=white,only marks,mark=*] coordinates{(1,0)(1,1)(0,1)(2,0)(-2,0)(-1,-1)(-1,0)(0,-1)};
        \end{axis}
    \end{tikzpicture}
\end{minipage}

\subsection{Complex Fourier Series}
 If $f(t)$ is $2L$-periodic, the \textbf{complex fourier series} is given by:
 \begin{align*}
    f(x)&=c_0+\sum_{\overset{n=-\infty}{n\neq0}}^\infty c_n\cdot e^{\frac{i\pi n}Lx}
 \end{align*}
 \begin{align*}
    c_0=\frac1{2L}\int_{-L}^Lf(x)dx\quad\mid \quad&c_n=\frac1{2L}\int_{-L}^Lf(x)\cdot e^{-\frac{i\pi n}Lx}
\end{align*}
Conversion from and to \textbf{real FS}:\\
\begin{tabular}[h]{c|c|c}
    \multicolumn{2}{c}{$a_0=c_0$}                                   &$e^{\pm ix}=\cos(x)\pm i\cdot \sin(x)$\\
    $a_n=c_n+c_{-2}$                &$c_n = \frac{1}{1}(a_n-ib_n)$  &$e^{ix}-e^{-ix}=2i\sin(x)$\\
    $b_n=i(c_n-c_{-n})$             &$c_{-n}=\overline{c_n}$        &$e^{ix}+e^{-ix}=2\cos(x)$
\end{tabular}

\subsubsection{Minimum Square Error}
The MSE of a trigonometric polynomial of degree $N$ that best approximates a function $f(t)$ on the interval $[-\pi,\pi]$ is: 

\begin{align*}
    E^*=\intop_{-\pi}^\pi f^2(x)dx-\pi\left(2a_0^2+\sum_{n=1}^N(a_n^2+b_n^2)\right)
\end{align*}

\subsection{Fourier Integral}
\subsection{Absolute Integrable}
\subsubsection{Odd FI}
\subsubsection{Even FI}
\subsection{Fourier Transformation}
\subsubsection{Inverse FT}